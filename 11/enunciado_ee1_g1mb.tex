\documentclass[12pt]{article}

\usepackage[a4paper]{geometry}

\usepackage[utf8]{inputenc}

\usepackage{hyperref}

% \usepackage{graphics} % Or
% \usepackage{graphicx} %% and then \includegraphics[scale=0.5,...]{filename}

\usepackage[spanish]{babel}
%\usepackage[spanish,english]{babel}

\usepackage{listings}

\lstset{language=Java,basicstyle=\tt}

% \usepackage{url}

% \usepackage{fancybox}
% \usepackage{fancyheadings}

\usepackage{courier}
\usepackage{fourier}

\title{
  Ejercicio Evaluable 1\\
  Filtrar datos de una pila
}
\author{Programación II\\G-1MB\\Escuela Técnica Superior de Ingenieros Informáticos\\Universidad Politécnica de Madrid}
\date{3 de abril de 2018\\\small{(Ángel Herranz)}}

\begin{document}
\maketitle

%% \tableofcontents

\begin{description}
\item[Objetivo:] El objetivo de este ejercicio entregable es la familiarización del alumno
  con las clases y objetos de Java y en concreto con el tipo abstracto de datos de las pilas.
\item[Entrega:] El ejercicio se entregará a través de moodle.
\item[Grupos:] El ejercicio se podrá realizar en grupos de dos alumnos o de forma
  individual. Se debe incluir al comienzo de cada fichero el/los número/s de matrícula
  de el/los autor/es del ejercicio.
\end{description}

\section{Enunciado del ejercicio}

Tu trabajo consiste en:
\begin{enumerate}
\item Implementar una clase que representa el tipo abstracto de datos de pilas acotadas con el siguiente comportamiento:
  \begin{itemize}
  \item El constructor admite el parámetro que indica la capacidad máxima de la pila.
  \item El método \lstinline{push} apila un string en la cima de la pila.
  \item El método \lstinline{pop} desapila el string de la cima de la pila.
  \item El método \lstinline{top} devuelve la cima de la pila (sin desapilarla).
  \item El método \lstinline{isEmpty} indica si la pila está vacía.
  \item El método \lstinline{isFull} indica si la pila está llena.
  \end{itemize}
\item Implementar un programa que lee un texto de la entrada estándar
  y filtra aquellas palabras de longitud superior a tres caracteres
  utilizando pilas acotadas.
\end{enumerate}

\section{Código de apoyo}

Se ofrece un fichero \url{FiltrarPalabras.java} que contiene:
\begin{itemize}
\item La función \lstinline{main}.
\item La clase \lstinline{StringStack}.
\end{itemize}

En el fichero hay indicaciones \textbf{TODO} (\emph{por hacer}) para que completes lo necesario y lograr que tu entrega sea correcta.

\section{Valoración de la entrega}

Tu programa tiene que dar esa salida para la entrada indicada. Además se comprobará:
\begin{itemize}
\item Que la implementación de las pilas acotadas es la de un tipo abstracto de datos: los atributos son privados, no hay más operaciones que las necesarias.
\item Que la implementación de las pilas acotadas respeta el significado de una pila (\emph{last-in first-out}).
\item Que el programa hace lo que se exige. Para ello se entregan dos ficheros más:
  \begin{itemize}
  \item Un fichero de texto \url{el_quijote.txt} que se usará como
    ejemplo de datos de entrada.
  \item Un fichero de texto \url{el_quijote.ans} que contiene la salida
    esperada por tu programa.
  \end{itemize}
\end{itemize}

\section{Consideraciones extra}

La entrega puede ser realizada por cualquiera de los dos alumnos del grupo, pero
no por los dos a la vez.

\end{document}

%%% Local Variables: 
%%% mode: latex
%%% TeX-master: t
%%% TeX-PDF-mode: t
%%% ispell-local-dictionary: "castellano"
%%% End: 
